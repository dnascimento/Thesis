%!TEX root = ../../tese.tex
%!TEX encoding = UTF-8 Unicode

\begin{resumo}
O número de aplicações instaladas usando o modelo Plataforma como Serviço (PaaS) tem aumentado. Apesar dos mecanismos de controlo de segurança implementados pelos operadores de serviços de computação em nuvem é expectável que estas aplicações sejam afectadas por intrusões. Neste documento introduzimos um novo serviço de recuperação de intrusões designado por Shuttle. O Shuttle permite que os operadores ofereçam um serviço através do qual os seus clientes podem recuperar de intrusões às suas aplicações.

A nossa abordagem permite inverter as alterações ao estado da aplicação provocadas por intrusões, sem comprometer o efeito de operações legítimas que ocorram após a intrusão. A abordagem de gravar e re-executar é combinada com a elasticidade oferecida pelo modelo de computação em nuvem para recuperar de intrusões a aplicações instaladas em várias instâncias e suportadas por uma base de dados distribuída. Para realizar a recuperação, o serviço carrega uma cópia da base de dados, gravada antes da intrusão ocorrer, e repete os pedidos posteriores, tão em paralelo quanto possível, enquanto processa novos pedidos.

A avaliação experimental realizada no serviços de computação na nuvem da Amazon (AWS) demonstra que o Shuttle é capaz de repetir 1 milhões de pedidos em aproximadamente 10 minutos e que é possível duplicar o número de pedidos repetidos por segundo aumentando o número de servidores de 1 para 3.

%Neste documento introduzimos um novo serviço para plataforma como serviço (Platform as a Service - PaaS). Através deste serviço, os fornecedores de serviços de computação em núvem permitem que os seus clientes recuperem de intrusões realizadas às suas aplicações. A motivaçã deste trabalho é o número crescente de intrusões e de aplicações criticas nos sistemas de computação em núvem e o desenvolvimento do modelo PaaS. O serviço proposto, designado de Shuttle, remove as intrusões a aplicações instaladas em PaaS. O serviço recupera de intrusões devido a vulnerabilidades de software, pedidos corrompidos dos utilizadores e suporta a actualização do sistema para correcção de vulnerabilidades.

%Shuttle recupera de intrusões a aplicações instaladas em multiplas instâncias e suportadas por uma base de dados. Para recuperar a integridade da aplicação, o serviço carrega uma snapshot da base dados anterior à intrusão e repete os pedidos, posteriores ao instante do snapshot, em paralelo, utilizando multiplos clientes. Shuttle usa a elasticidade e o modelo pago pelo que usar dos sistemas \ac{PaaS} para criar novas instancias com a aplicação ou base de dados para atender o fluxo de pedidos do processo de repetição, reduzindo o periodo de recuperação. Mais, o serviço usa \ac{PaaS} para instalar novas versões da aplicação para resolver falhas de software anteriores e remover instâncias corrompidas. O Shuttle não requer que a aplicação fique indisponível durante o processo de recuperação porque é utilizado um sistema de branching \hl{como traduzo?!}. O Shuttle é um serviço que está disponível aos programadores de aplicações \ac{PaaS} sem necessitar de instalação ou configuração.

\end{resumo}
\vspace{-2cm}
\begin{palavraschave}
\vspace{-1cm}
\begin{itemize}
\item Recuperação de Intrusões
\item Tolerância de Intrusões
\item Dependência
\item Computação em Nuvem
\item Platforma como Serviço
\item Sistemas de Bases de Dados Distribuídas
\end{itemize}
\end{palavraschave}
\clearpage
\thispagestyle{empty}
\cleardoublepage
