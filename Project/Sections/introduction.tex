%!TEX encoding = UTF-8 Unicode
\section{Introduction}\label{sec:Introduction}

%PaaS
Platform as a Service (PaaS) is a cloud computing model that supports automated configuration and deployment of applications. While the Infrastructure as a Service (IaaS) model is being much used to obtain computation resources and services on demand \cite{Lenk2009,Armbrust2009}, PaaS aims to reduce the cost of software deployment and maintenance abstracting the underlying infrastructure. This model defines an environment for deployment and execution of applications in \textit{containers}. PaaS containers support the software, e.g., Apache Tomcat \cite{tomcat}, where developers can deploy their applications. Containers can be bare metal machines, virtual machines or based on processes and resources isolation mechanisms, such as Linux control groups \cite{Menage2007}. Most of PaaS systems provide a set of programming and middleware services to support application design, implementation and maintenance. Examples of these services are load-balancing, automatic server configuration and storage APIs. These services aim to provide a well tested and integrated development and deployment framework \cite{Vaquero2011}. PaaS systems are provided either by cloud providers \cite{AmazonElasticBeanstalk,GoogleAppEngine,Heroku,OpenShift} or open source projects \cite{Appscale,Cloudfoundry,ApacheStratos}. Besides natural metrics such as cost and performance, the success of PaaS systems will also be established by their features, for instance recovery services.\\


%The problem
The number of critical and complex applications in the cloud, particularly using PaaS, is increasing rapidly. Most of the customers' and companies' critical applications and valuable informations are migrating to cloud environments. Consequently, the value of the deployed applications is superior. As a result, the risk of intrusion is higher because the exploitation of vulnerabilities is more attractive and profitable. An intrusion happens when an attacker exploits a vulnerability successfully. Intrusions can be considered as faults. Faults may cause system failure and, consequently, application downtime which has significant business losses \cite{Patterson2002a}. Recovery services are needed to remove the intrusion and restore the correct application state.\\

%Alternatives: Prevention
Prevention and detection of malicious activities are the priorities of a substantial number of security processes. However, preventing vulnerabilities by design is not enough because software tends to have flaws due to complexity and budget/time constraints \cite{Charette2005}. More, attackers can spend years developing new ingenious and unanticipated attack methods having access to what protects the application. On the opposite side, guardians have to predict new methods to mitigate vulnerabilities and to solve attacks in few minutes to prevent intrusions. On the other hand, a vast number of redundancy mechanisms and Byzantine fault tolerance protocols are designed for random faults but intrusions are intentional malicious faults designed by human attackers to leverage protocols and systems vulnerabilities. Most of these techniques do not prevent application level attacks or usage mistakes. If attackers use a valid user request, e.g., stealing his credentials to delete his data, most of replication mechanism will spread the corrupted data. Therefore, the application integrity can be compromised and the intrusion reaches its goal bringing the system down to repair.\\

%Why Replay on PaaS (Our solution)?
The approach followed in this work consists in recovering the applications state when intrusions happen, instead of trying to prevent them from happening. Intrusion recovery does not aim to substitute prevention but to be an additional security mechanism. Similarly to fault tolerance, intrusion recovery accepts that faults occur and have to be processed.

Our goal is to design, implement and evaluate \textit{Shuttle}, an intrusion recovery service for PaaS systems. Shuttle recovers from intrusions in the software domain due to software flaws, corrupted requests, input mistakes, corrupted data. It also supports corrective and preventive maintenance of PaaS applications. Shuttle aims to recover the integrity of applications without compromising their availability. Shuttle cannot avoid information leaks, so confidentiality is out of the scope of this work. Normally, recovering from intrusions requires extensive human intervention to remove the intrusion effects and restore the application state. For example, most of full-backup solutions revert the intrusion effects but require extensive administrator effort to replay the legitimate actions. Therefore, we believe that a service which removes the intrusion effects and restores the applications integrity, without exposing a downtime during the process, is a significant asset to the administrators of PaaS applications.

The rapid and continuous decline in computation and storage costs in public cloud providers makes affordable to store user requests, to use database checkpoints and to replay previous user requests. We will use these mechanisms to recover from intrusions. Despite the time and computation demand to replay the user requests, cloud pricing models provide the same cost for 1000 machines during 1 hour than 1 machine during 1000 hours \cite{Armbrust}. Shuttle leverages the resources elasticity in PaaS environments to record and replay non-malicious user requests. In order to remove the corrupted state, Shuttle loads a previous backup and creates new containers using intrusion-free images. The images may include software updates, which may fix previous flaws. Then, Shuttle replays the user requests in parallel, using multiple machines, to recreate an intrusion-free application state. 

Shuttle only requires a valid input record and an intrusion-free container image to recover the applications state integrity, i.e., a state that allows the applications to behave according to their specification. Shuttle will be available for developers of PaaS applications usage without installation, configuration. In addition, their applications source code will remain identical, or remain unmodified as much as possible.\\

%Contributions
In this work, we propose, to the best of our knowledge, the first intrusion recovery service for PaaS applications. We also are amongst the first to consider recovery in distributed database applications, where the data may be stored in multiple database instances. We suggest, to the best of our knowledge, the first intrusion recovery system using request replay which takes into consideration applications running in various containers. We incorporate the possibility of renewing the image of containers to remove intrusions. Moreover, only few works have been presented that accomplish intrusion recovery without service downtime.\\

%Road map
The remainder of the document is structured as follows. Section \ref{sec:Goals} explains the goals and expected results of our work. Section \ref{sec:RelatedWork} presents the fundamental concepts and previous intrusion recovery proposals. Section \ref{sec:Architecture} describes briefly the architecture of PaaS systems and the proposed architecture for intrusion recovery service. Section \ref{sec:Evaluation} defines the methodology which will be followed in order to validate the proposed service. Finally, Section \ref{sec:Schedule} presents the schedule of future work and Section \ref{sec:Conclusion} concludes the document.


