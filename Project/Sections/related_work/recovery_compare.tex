\subsection{Evaluation of Technologies}
\label{sec:Discuss}
The previous subsections describe various services that use different approaches to recover from intrusions. The level of abstraction outlines the recorded elements. Intrusion recovery services at the operating system level record the system calls and the file system. At the database level they track the transactions. At the application level they track the transactions, requests and code execution (Table \ref{tab:storingTracking}). The log at operating system level is more detailed but may obfuscate the attack in false positive prevention techniques. At database and application level, the log is semantically rich but does not track low level intrusions.\\


Most of services require the administrator to identify the initial set of corrupted actions or objects (Table \ref{tab:storingTracking}). The administrator may be supported by an intrusion detection system (IDS). The alternative, proposed in Warp \cite{warp}, tracks the requests which invoked modified code files. The identification of the actions or objects can incur on false positives and negatives due to administrator mistakes. Tracking the invoked code files requires to change the interpreter (JVM, Python, PHP) but avoids false positives and negatives.\\

The taint algorithm, which determines $D_{tainted}$, is performed statically using the original execution dependencies. The dependencies are recorded in a dependency graph or determined dynamically replaying the legitimate actions which have a different input in replay phase than in original execution. The later reflects the changes during the replay phase, therefore it is clearly better but requires to store the input of every action. An alternative proposed by Undo for Operators \cite{undoForOperators} sorts the requests, using the knowledge of the application protocol, and then loads a previous snapshot and replays the legitimate requests. This approach is possible because \textit{every} legitimate request is replayed with a known order. The tradeoff between perform taint propagation via replay or replay every request is equivalent to a trade-off between storage and computation. In the first, the system must store every version of each data item while in the later system only stores the versions of each data item periodically. In the worst case, if all data entries are tainted by the intrusion, the first approach will replay the same number of requests as the second. 

Warp \cite{warp} supports application source code changes tracking the original code invocations and comparing output when the application functions are re-invoked. Undo for Operators \cite{undoForOperators} support application changes using application dependent compensating actions to resolve conflicts. Goel \textit{et Al} \cite{Akkus2010} do not support code changes because the used taint via replay mechanism stops if the action input is similar. The remaining proposals do not support code change. 


\begin{center}
  \rowcolors{1}{}{lightgray}
\begin{table}[h]
  \hspace*{-2cm}
\begin{tabular}{|l|m{4.5cm}|m{3cm}|m{3cm}|C{1.8cm}|}
\hline
\textbf{System} & \textbf{Data logged}		& \textbf{Intrusion \newline identification data} & \textbf{Taint \newline mechanism} & \textbf{Supports action changes} \\ \hline
\cite{taser}Taser			  		& System call inputs 			& Tainted or intrusion source objects 	&Graph						& \xmark		\\ \hline
\cite{retro} \cite{dare} RETRO,Dare	& System call inputs and outputs	& Intrusion objects or actions   &Taint via replay				& \xmark		\\ \hline
\cite{itdb} ITDB						& Read/write sets using SQL parsing & Intrusion objects &Set expansion (graph)		& \xmark \\ \hline
\cite{phoenix} Phoenix					& Read/write sets using DBMS modification & Intrusion actions & Graph 				& \xmark		\\ \hline
\cite{Akkus2010} Goel \textit{et al.}	& User, session, request, code execution and database rows 		&  Intrusion requests &  Graph and taint via replay & \xmark \\ \hline
\cite{warp} \cite{aire} Warp,Aire		& Client side browser interactions, requests, user sessions, invoked PHP files &Requests which invoked the modified source code files & Taint via replay & \cmark	\\ \hline
\cite{undoForOperators} Undo for Operators	&Requests		&Intrusion requests        &Application protocol dependencies   &	\cmark		\\ \hline
Shuttle	(this work)							& User requests, session and read/write set using DBMS changes & Intrusion requests  & Graph and taint via replay & \cmark \\ \hline

\end{tabular}
 \caption{Summary of storing and intrusion tracking options}
  \label{tab:storingTracking}
\end{table}
\end{center}


\begin{center}
  \rowcolors{1}{}{lightgray}
\begin{table}[h]
  \hspace*{-2cm}
\begin{tabular}{|l|m{2.5cm}|m{4.5cm}|m{3.5cm}|C{1.6cm}|C{1.7cm}|}
\hline
\textbf{System} & \textbf{Previous state recovery}		& \textbf{Effect removal} & \textbf{Replay phase} & \textbf{Runtime recovery} & \textbf{Externally consistent} \\ \hline
\cite{taser}Taser			  		& Snapshot 	& load legitimate entries value from a snapshot & No &  &  \\ \hline
\cite{retro} \cite{dare} RETRO,Dare	& Snapshot 	& load legitimate entries value from a snapshot & Tainting via replay & &  \\ \hline
\cite{itdb} ITDB						& Transaction compensating & Compensate tainted transactions & No & \cmark & \\ \hline
\cite{phoenix} Phoenix					& Row versioning & Abort tainted transactions & No & 	&		\\ \hline
\cite{Akkus2010} Goel \textit{et al.}	& Transaction compensating & Compensate tainted transactions & Tainting via replay & & \\ \hline
\cite{warp} \cite{aire} Warp,Aire		& Row versioning & Load previous entry version & Tainting via replay & \cmark & \cmark	\\ \hline
\cite{undoForOperators} Undo for Operators	& Snapshot & load snapshot & Replay all requests sorted by application semantics & & \cmark		\\ \hline
Shuttle	(this work)								& Snapshot & Load snapshot & Replay tainted requests sorting by original execution read/write sets & \cmark  & \cmark \\ \hline

\end{tabular}
 \caption{Summary of state recovery options}
  \label{tab:recovery}
\end{table}
\end{center} 


At recovery phase, the intrusion recovery services remove the intrusion effects and recover a consistent state  (Table \ref{tab:recovery}). Three of the possible options to remove the intrusion are: snapshot, compensating and versioning. Versioning, which is a per-entry and per-write snapshot, is finer-grained and allows the reading of previous versions without replay the actions after the snapshot. However, the storage requirements to keep the versions of every entry in a large application can be an economic barrier. The usage of actions compensation requires the knowledge of the actions that revert the original actions.

To recover a consistent state, intrusion recovery services should replay the legitimate actions dependent from the intrusion actions (Section \ref{subSec:RecoveryModels}). While Undo for Operators \cite{undoForOperators} propose to replay all legitimate user requests sorted by an application-dependent algorithm, the remaining services, which perform replay, use tainting via replay to selectively replay only the dependent actions. The late approach may hide indirect dependencies \cite{Xie2008} but recovers faster. Undo for Operators \cite{undoForOperators}, replays every request generating conflicts that must be solved in order to achieve a consistent state. Warp \cite{warp} queues the conflicts for later solving by users. The proposals \cite{itdb,phoenix,warp,aire} support runtime recovery (Section \ref{subSec:RecoveryModels}). This characteristic is required to support intrusion tolerance but allows intrusion to spread during the recovery period. To prevent that, ITDB \cite{itdb} denies the access to tainted data. However, it compromises the availability during the recovery period.\\

