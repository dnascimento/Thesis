%!TEX root = ../paper.tex
%!TEX encoding = UTF-8 Unicode

\section{Related Work}
\label{sec:related_work}

Shuttle is an intrusion recovery system based on log replay. While this approach has been applied in operating systems \cite{taser,retro,dare}, databases \cite{itdb,phoenix}, and web services \cite{warp,goel,aire}, our system is the only one that recovers from intrusions in applications deployed on \ac{PaaS} platforms with multiple servers. The closest works to Shuttle are Aire \cite{aire}, Warp \cite{warp}, Goel \cite{goel}, and Undo for Operators (UO) \cite{undoForOperators}, although none of them does recovery in cloud environments.

Shuttle's full replay approach is motivated by UO \cite{undoForOperators}. UO removes the intrusion effects using a snapshot and replaying every request after the snapshot instant.\LONG{However, the primary source of difference between the two approaches is their target applications.} UO considers monolithic applications, which are instantiated in the paper as an email server. Shuttle, on the contrary, considers a \ac{PaaS} platform with both application server and database instances, supporting scalable applications of several kinds\LONG{(e.g., Q\&A, social networks, shared editors, etc.)}. UO sorts requests using knowledge of the application protocol. Developers must define, for each type of application request, the order between requests and their capability to be executed in parallel. In contrast, Shuttle uses the dependency graph to create clusters of request to execute in parallel and sorts the requests using the start-end order and database accesses.

%mpc: cuidado: nao e' et. all mas et al. (abreviatura de "et alia")

Goel \emph{et al.} \cite{goel} proposes a solution to recover from intrusion in web applications. It uses a modified PHP interpreter to determine the tainted requests. Goel reverts the effect of tainted requests applying compensating transactions on the current state of the database. %Unlike Shuttle, it does not replay requests. Therefore, effects of legitimate actions, which have been tainted, are lost.

Warp \cite{warp} helps the administrators to retroactively patch security vulnerabilities. It stores every version of each data item and the version read by each request. It also captures the browser-side input at DOM level using a browser plugin and modifies the code interpreter to track the code files invoked by each request. Requests that invoked code files modified by the patch are considered tainted. Warp loads the version of the tainted data items and repeats these requests using a server-side browser. Forward requests are also replayed while their inputs are different from the ones at first execution. Most of previous solutions store every data item version or action input and output \cite{warp,aire}. Shuttle incurs on smaller storage overhead because it gets the data item version from a snapshot and replays every legitimate request at least until the next snapshot, where the data item value is known.

%Get previous values
There have been different approaches to remove the intrusion effects: Goel \emph{et al.} use transaction compensation to create a snapshot, Warp stores every data item version and UO uses snapshots. While the first does not remove the effects of unknown actions, the second requires a considerable storage overhead. We implemented a snapshot mechanism designed for distributed databases. While UO overwrites the current state copying an old snapshot, we use the \textit{branch path} mechanism to select the accessible snapshots for a given request without copying data.

%runtime recovery:
Warp supports recovery in runtime using two fields in the database table, whereas Aire uses a branching mechanism to permit recovery on loosely coupled web services. We use a branching mechanism to support runtime recovery allowing various recovery processes simultaneously. 

%inconsistency
Goel \emph{et al.} does not address external consistency issues. Warp \cite{warp} detects inconsistencies in responses and replays the user interaction using a browser. UO uses compensating actions based on protocol-specific knowledge. We propose an API that the application developers can use to deal with inconsistencies.

