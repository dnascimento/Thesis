%!TEX root = ../paper.tex
%!TEX encoding = UTF-8 Unicode

\begin{abstract}
The number of applications being deployed using the Platform as a Service (PaaS) cloud computing model is increasing. Despite  the security controls implemented by cloud service providers, we expect intrusions to harm such applications. We present Shuttle, a novel intrusion recovery service. Shuttle recovers from intrusions in applications deployed in PaaS platforms.
%
Our approach allows undoing changes to the state of PaaS applications due to intrusions, without loosing the effect of legitimate operations performed after the intrusions take place. We combine a record-and-replay approach with the elasticity provided by cloud offerings to recover applications deployed on various instances and backed by distributed databases. The service loads a database snapshot taken before the intrusion and replays subsequent requests, in parallel as much as possible, while continuing to execute incoming requests. 
%Shuttle not only avoids application downtime during recovery, but also allows customers to deploy new application versions to fix previous software flaws.
%
We present an experimental evaluation of Shuttle on Amazon Web Services. We show Shuttle can replay 1 million requests in 10 minutes and that it is possible to duplicate the number of requests replayed per second by increasing the number of application servers from 1 to 3. 

%The increasing number of intrusions and critical applications deployed in cloud systems requires an approach to recover from intrusions. We present Shuttle, a novel service \hl{approach?} for \acf{PaaS} systems that gives cloud providers the capability to allow their customers to recover their applications from intrusions. Through exploiting the novel \ac{PaaS} architecture \hl{model?}, the proposed service removes security intrusions due to software flaws or corrupted user requests and supports corrective and preventive maintenance of applications deployed in \ac{PaaS}.

%We combine the record-and-replay approach with the elasticity and pay-per-usage model of \ac{PaaS} to recover from intrusion in applications deployed on various instances and backed by distributed database. We propose to load a previous database snapshot and replay the requests in parallel using a set of cloud instances to reduce the recovery period. Shuttle not only avoids application downtime during the recovery but also allows customers to deploy new application versions to fix previous software flaws.

%Through our evaluation, we demonstrate that Shuttle incurs negligible \hl{acceptable?} performance overhead and that performing parallel replay using \hl{XXX} client instances and \hl{XXX} new application instances and \hl{XXX} database instances can reduce the recovery period by \hl{XXX}\%. We also show Shuttle can recover an application with \hl{XXX} requests, storing \hl{XXX} of metadata and replaying the requests in \hl{xxx} seconds using \hl{xxx} cloud instances.

\end{abstract}
