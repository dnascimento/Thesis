%!TEX root = ../paper.tex
%!TEX encoding = UTF-8 Unicode

\section{Background and Related Work}\label{sec:background}
%Como e que as apliccaoes sao recuperadas, de modo generico e com o related work incluido
%This section formalizes several approaches to perform intrusion recovery and presents the main related work. 
An application execution is modeled as a set of actions $A$ on a set of objects $D$. Actions are described by operations (read, write, others more complex), the value(s) read/written, and a timestamp (which defines the order of the actions). Each object has a state (or value) and a set of operations that can modify it. We define $A_{intrusion}$ as the subset of actions of $A$ whereby the attacker compromises the application during the intrusion and $A_{legal}$ as the subset of legitimate actions in $A$, i.e., $A_{legal} = A \backslash A_{intrusion}$.

Intrusion recovery services aim to remove the effects of malicious actions setting the application state to a state set only by legitimate actions. 
%The recovered application state shall respect the application specification (correctness) \cite{Aviz}.
%
A \textit{backup} mechanism is a basic recovery service that can set objects to the state they had before an intrusion began. The new state excludes the effects of the attacker's actions, but also the effects of any legitimate actions performed after the backup was done. This second aspect is undesirable, so recent intrusion recovery systems aim to avoid it. This is the case in the closest works to ours: Aire \cite{aire}, Warp \cite{warp}, Goel \cite{Akkus2010}, and Undo for Operators (UO) \cite{undoForOperators}. None of these works handles recovery in cloud environments.

%tainted
An action is considered \textit{tainted} at a certain instant if it depends from a malicious or tainted action, i.e., it reads an object written by a malicious or tainted action (called a \textit{tainted object}). Since actions are contaminated by malicious actions through objects, to remove the state written by tainted actions it is necessary but not sufficient to obtain the state produced by the legitimate actions. The tainted actions need to be re-executed to read different values (not imposed by the attacker) and have a different execution. 


%Get previous values
%DN: Deixei cair a definicao de version, assumo que os leitores sabem o que e uma versao
%A \textit{version} is a snapshot of a single object value at an instant $t$. Versions can be recorded with the sequence of actions that write the objects before the instant $t$.
There have been different approaches to remove intrusion effects removing the object versions written by malicious actions: Goel \cite{Akkus2010} use transaction compensation (invert transactions' effects) to create snapshots, Warp loads previous data item version and UO loads snapshots. The first approach does not remove the effects of unknown actions, the second requires a considerable storage overhead. We implemented a snapshot mechanism designed for distributed databases. %While UO overwrites the current state copying an old snapshot, we use a \textit{branching} mechanism to select the accessible snapshots for a given request without copying data.

% That generic approach is unfeasible without the use of snapshots.
% The sequence of actions performed before the intrusion $A_{before} = A \backslash A_{after}$ can be long and each action takes non-null time to execute, so replaying $A_{before}$ may be unfeasible. Moreover, a log of all the actions executed may be too large.
% We define the subsets $D_{snapshot}(t)$ and $A_{snapshot}(t) : A_{snapshot}(t) \subset A$ as the subsets of objects and actions executed before the begin of a snapshot operation at instant \textit{t}. 
% %The snapshot operation copies the value of the object immediately or on the next write operation. 
% If the intrusion happens after $t$, then $A_{after} \cap A_{snapshot}(t) = \emptyset \implies (A_{intrusion} \cup A_{tainted}) \cap A_{snapshot}(t) = \emptyset$, i.e., the snapshot is not affected by the intrusion. For that reason, the service can replay only $A \backslash A_{snapshot}(t) \backslash A_{intrusion}$ using the object set $D_{snapshot}$ as base. 

There are two distinct replay approaches to update to a correct state (setting $D$ to $D_{recovered}$): selective replay and full replay. 
The \textit{selective replay} approach loads only the versions of the tainted objects, $D_{tainted}$, previous to the intrusion, instead of loading a previous version of every object \cite{taser,warp,Akkus2010}. Then, it replays only $A_{tainted}$, to update the objects in $D$. Further actions are also replayed if their input is different (tainting via-replay). 

The other approach, \textit{full replay} \cite{undoForOperators}, loads a snapshot previous to the intrusion moment and replays every non-malicious action executed after the snapshot operation. This approach is slightly simpler than the other, but takes longer to execute.  


%runtime recovery: 
%Warp supports recovery in runtime using two fields in the database table, whereas Aire uses a branching mechanism to permit recovery on loosely coupled web services. We use a branching mechanism to support runtime recovery allowing various recovery processes simultaneously. 

%inconsistency:  
%Goel does not address external consistency issues. Warp \cite{warp} detects inconsistencies in responses and replays the user interaction using a browser. UO uses compensating actions based on protocol-specific knowledge. We propose an API that the application developers can use to deal with inconsistencies.

%UO considers monolithic applications, which are instantiated in the paper as an email server. Shuttle, on the contrary, considers a \ac{PaaS} platform with both application server and database instances, supporting scalable applications of several kinds and does not require knowledge of the application protocol. 
















\LONG{
Recovery services have two distinct phases: record and recovery. The \emph{record phase} is the service usual state where the application is running and the service records the application actions. In order to perform replay, the application actions do not need to be idempotent but their re-execution must be deterministic (given the same initial state they produce the same final state). The record phase should record the actions input and the value of every non-deterministic behavior to turn their re-execution into a deterministic process. The \emph{recovery phase} can be subdivided in three: determining the affected actions and/or objects, removing these effects, and replaying the actions necessary to recover a consistent state. In this paper we present a recovery service that supports \textit{runtime recovery}, i.e., that allows the record and recovery phases to occur simultaneously.

}

\LONG{

% \hl{este explica o dependency graph. ou o related work sai e vem para aqui ou entao nao vale a pena ter um pagrafo a falar disto quando ha uma seccao} %mpc: ok, comentei
% Most intrusion recovery services record both the actions and  the objects they accessed  \cite{Akkus2010,itdb,warp}. Since the actions read and write objects from a shared set of objects $D$, we can establish dependencies between actions. Dependencies can be seen as an \textit{action dependency graph} or an \textit{object dependency graph}. The nodes of an action dependency graph represent actions and the edges indicate dependencies though shared objects. An object dependency graph establishes dependencies between objects through actions. Dependency graphs have been used to order the re-execution of actions \cite{undoForOperators}, get the sequence of actions affected by an object value change \cite{warp}, get the sequence of actions tainted by an intrusion \cite{Akkus2010} or resolve the set of objects and actions that caused the intrusion using a set of known tainted objects \cite{backtracker}. 

%A \textit{taint algorithm} aims to define the tainted objects $D_{tainted}$ from a set of malicious actions $A_{intrusion}$ or objects $D_{intrusion}$ using a dependency graph. This method is used by selective replay approaches. The \textit{taint propagation via replay} \cite{retro} algorithm begins with the set $D_{tainted}$ determined by the base taint algorithm \hl{o que significa?} and expands the set \hl{o que significa?} $D_{tainted}$. It is used to restore the values of $D_{intrusion} \cup D_{tainted}$ and to replay only the legal actions that output $D_{intrusion} \cup D_{tainted}$ during the original execution. Then it replays the actions dependent from $D_{intrusion} \cup D_{tainted}$, updating their output objects. While the forward actions have different input, they are also replayed and their outputs are updated. 

% Dependencies are established during the record phase or at recovery time using the objects and actions. The level of abstraction influences the record technique and the dependency extraction method. The abstraction level defines the recoverable intrusions: operating system \cite{taser,retro}, database \cite{itdb,phoenix}, and application \cite{Akkus2010,warp,aire}. In the next sections, we present Shuttle, an intrusion recovery service which recovers from intrusions using the dependencies established at database and application level.

%mpc: comentei pois nao percebi a mensagem que queriamos passar e nao parece exacto
%Some intrusion recovery systems \cite{taser,itdb,phoenix} attempt to do it \textit{replacing} the value of the tainted objects by a previous value. These systems keep the objects written by legitimate actions unmodified. %rever o taser, itdb e phoenix para confirmar

}




\LONG{
%mpc: juntei 2 paragrafos e limpei as varias repeticoes das mesmas ideias
% Consider a hypothetical application execution. At a certain point in time after an intrusion, the application is stopped and the sequence of actions executed ($A$) is purged of the actions in $A_{intrusion}$. Then, the state is rewind to the beginning of the application execution and only the actions in $A_{recovered} = A \backslash A_{intrusion} = A_{legal}$ are re-executed. That re-execution is intrusion-free as $A \cap A_{intrusion} = \emptyset \implies A_{tainted} = \emptyset$. The set of tainted actions of the first execution that are not in $A_{intrusion}$ are re-executed but read different values (not imposed by the attacker) and have a different execution. The replay process restores the application to a correct state $D_{recovered}$, which is not compromised by the intrusion.
}

%mpc: movi a definicao de version para o parag anterior porque fazia la' falta
%We define a compensating action as an action that reverts the effects of a original action, for instance writing a previous value. A compensation process can obtain a previous snapshot or version. For this propose, we define the sequence Acompensation(t) as the compensation of Aposteriori(t), the sequence of actions after instant t. The compensation process applies the sequence of compensating actions Acompensation(t) on the current version of the objects, in reverse order, to obtain a previous snapshot or version.





%%%%%%%%%%%%%%%%%%%%%% Versao do professor %%%%%%%%%%%%%%%%%%%%%%%%%%%%%
% This section formalizes several approaches to perform intrusion recovery and presents the main related work. 
% %
% An application execution is modeled as a set of actions $A$ on a set of objects $D$. Actions are described by operations (read, write, others more complex), the value(s) read/written, and a timestamp (which defines the order of the actions). Each object has a state (or value) and a set of operations that can modify it. We define $A_{intrusion}$ as the subset of actions of $A$ whereby the attacker compromises the application during the intrusion, $A_{after}$ as the subset of actions that began after the intrusion began (including the first action of the intrusion), and $A_{legal}$ as the subset of legitimate actions in $A$, i.e., $A_{legal} = A \backslash A_{intrusion}$.

% Intrusion recovery services aim to remove the effects of malicious actions setting the application state to a state set only by legitimate actions. 
% %The recovered application state shall respect the application specification (correctness) \cite{Aviz}.
% %
% A \textit{backup} mechanism is a basic recovery service that can set objects to the state they had before an intrusion began. The new state excludes the effects of the attacker's actions, but also the effects of any legitimate actions performed after the backup was done. This second aspect is undesirable, so recent intrusion recovery systems aim to avoid it.
% %
% This is the case in the closest works to ours: Aire \cite{aire}, Warp \cite{warp}, Goel \cite{Akkus2010}, and Undo for Operators (UO) \cite{undoForOperators}. None of these works handles recovery in cloud environments.

% An action is considered \textit{tainted} at a certain instant if it is one of the attacker's malicious actions or if it reads an object written by a tainted action (called a \textit{tainted object}). Since actions are contaminated by malicious actions through objects, to remove the state written by tainted actions it is necessary but not sufficient to obtain the state produced by the legitimate actions. 

% %mpc: comentei pois nao percebi a mensagem que queriamos passar e nao parece exacto
% %Some intrusion recovery systems \cite{taser,itdb,phoenix} attempt to do it \textit{replacing} the value of the tainted objects by a previous value. These systems keep the objects written by legitimate actions unmodified. %rever o taser, itdb e phoenix para confirmar


% %mpc: juntei 2 paragrafos e limpei as varias repeticoes das mesmas ideias
% Consider a hypothetical application execution. At a certain point in time after an intrusion, the application is stopped and the sequence of actions executed ($A$) is purged of the actions in $A_{intrusion}$.
% % i.e., the intrusion actions $A_{intrusion}$ are not executed. 
% %
% Then, the state is rewind to the beginning of the application execution and only the actions in $A_{recovered} = A \backslash A_{intrusion} = A_{legal}$ are re-executed. That re-execution is intrusion-free as 
% %mpc: D_intrustin e D_tainted nem sequer foram definidos
% %$A \cap A_{intrusion} = \emptyset \implies D_{intrusion} = \emptyset, A_{tainted} = \emptyset \implies D_{tainted} = \emptyset$. 
% $A \cap A_{intrusion} = \emptyset \implies A_{tainted} = \emptyset$. 
% %Since the malicious actions were removed, the state, $D$, would not have values imposed by $A_{intrusion}$. 
% %For this reason, the sequence of tainted actions $A_{tainted}$ would be empty. 
% The set of tainted actions of the first execution that are not in $A_{intrusion}$ are re-executed but read different values (not imposed by the attacker) and have a different execution. 
% %Therefore, if $A_{intrusion}$ and $D_{intrusion}$ are removed, then $A_{tainted}$ should be \emph{replayed} because the actions of $A_{tainted}$ are not contaminated by malicious data during their re-execution. 
% The replay process restores the application to a correct state $D_{recovered}$, which is not compromised by the intrusion.

% That generic approach is unfeasible without the use of snapshots.
% The sequence of actions performed before the intrusion $A_{before} = A \backslash A_{after}$ can be long and each action takes non-null time to execute, so replaying $A_{before}$ may be unfeasible. Moreover, a log of all the actions executed may be too large.
% We define the subsets $D_{snapshot}(t)$ and $A_{snapshot}(t) : A_{snapshot}(t) \subset A$ as the subsets of objects and actions executed before the begin of a snapshot operation at instant \textit{t}. 
% %The snapshot operation copies the value of the object immediately or on the next write operation. 
% If the intrusion happens after $t$, then $A_{after} \cap A_{snapshot}(t) = \emptyset \implies (A_{intrusion} \cup A_{tainted}) \cap A_{snapshot}(t) = \emptyset$, i.e., the snapshot is not affected by the intrusion. For that reason, the service can replay only $A \backslash A_{snapshot}(t) \backslash A_{intrusion}$ using the object set $D_{snapshot}$ as base. 

% There are two distinct replay approaches to update the set of object $D$ to $D_{recovered}$, 
% %because of changes in the execution of $A_{tainted}$
% selective replay and full replay. 
% A \textit{version} is a snapshot of a single object value at an instant $t$. Versions can be recorded with the sequence of actions that write the objects before the instant $t$.
% The \textit{selective replay} approach loads only the versions of the tainted objects, $D_{tainted}$, previous to the intrusion, instead of loading a previous version of every object \cite{taser,warp,Akkus2010}. Then, it replays only the legitimate actions, which were tainted, $A_{tainted} \backslash A_{intrusion}$, to update the objects in $D$. The state of the objects in $D_{legal} \backslash D_{tainted}$ remains unmodified. 
% The other approach, \textit{full replay} \cite{undoForOperators}, loads a snapshot previous to the intrusion moment and replays every action in $A \backslash A_{snapshot}(t) \backslash A_{intrusion}$. This approach is slightly simpler than the other, but in general takes longer to execute. 
